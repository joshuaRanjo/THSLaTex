


%
%-- specify related packages
%

%
% \usepackage[utf8x]{inputenc}
%

\usepackage{apacite}           %-- APA style citation 
                               %-- refer to http://www.ctan.org/tex-archive/biblio/bibtex/contrib/apacite/

%
%  \usepackage{ucs}
%

\usepackage{changepage}

\usepackage{amsmath}           %-- American Math Society packages
\usepackage{amsfonts}
\usepackage{amssymb}

\usepackage{float}


\usepackage{graphicx}          %-- graphicx package needed for including figures in JPG or PNG format
 
%
%\usepackage{graphics}          %-- graphics related package (this was commented out) use when image is in EPS format
%

\usepackage{verbatim}          %-- this package allows you to have multiple lines of comments by
                               %-- example:
                               %   \begin{comment}
                               %        ...your text here...
                               %   \end{comment}  

\usepackage{color}             %-- allows use of color with text
                               %-- example:  \textcolor{red}{This is the colored text in red.}

\usepackage{url}  %-- allows use of URLs example: \url{https:\ccs1.dlsu.edu.ph}
% Added packages by team
\usepackage{booktabs}          %-- package from JASP statistics tool
\usepackage{pdfpages}          %-- allows use of PDF files in document
\usepackage{subcaption}
\usepackage[utf8x]{inputenc}
\usepackage{breakurl}
\renewcommand{\UrlFont}{\normalfont}

%
%-- set margins,  you may need to edit this for your own printer
%
\topmargin 0.0in
\oddsidemargin 0.0in
\evensidemargin 0.0in

\voffset 0.0in
\hoffset 0.5625in

\textwidth 5.75in
\textheight 8.5in


\parskip 1em
\parindent 0.25in

\bibliographystyle{apacite}            %-- use APA citation scheme

\hyphenation{ana-lysis know-ledge}     %-- LaTeX may not hyphenate correctly some words you use in your document
                                       %-- use \hyphenation to instruct LaTeX how to do it correctly, example above

\newcommand{\degree}{^{\circ}}         %-- use \newcommand to create your own "commands"
                                       %-- \newcommand works like the #define you learned in your COMPRO1 class

\newcommand{\etal}{et al.}
\newcommand{\figref}[1]{Figure \ref{#1}}
\newcommand{\appref}[1]{Appendix \ref{#1}}

\newenvironment{subs}
  {\adjustwidth{3em}{0pt}}
  {\endadjustwidth}


\newcommand{\thestitle}[1]{{\Large \textsc{#1}}}