%%%%%%%%%%%%%%%%%%%%%%%%%%%%%%%%%%%%%%%%%%%%%%%%%%%%%%%%%%%%%%%%%%%%%%%%%%%%%%%%%%%%%%%%%%%%%%%%%%%%%%
%
%   Filename    : abstract.tex 
%
%   Description : This file will contain your abstract.
%                 
%%%%%%%%%%%%%%%%%%%%%%%%%%%%%%%%%%%%%%%%%%%%%%%%%%%%%%%%%%%%%%%%%%%%%%%%%%%%%%%%%%%%%%%%%%%%%%%%%%%%%%

\chapter{Conclusion}
\label{sec:results }

\section{Conclusion}
Two learning outcomes (LOs) were selected from the Analytical Geometry content found in the Department of Education’s curriculum guide for Senior High School STEM Course, as the target learning outcomes. Concepts regarding conic sections were focused on, specifically recognizing equations and important characteristics of different  of conic sections and solving situations involving conic sections.

Given these target learning outcomes, LO mechanics were designed to be used as game mechanics. These LO mechanics were integrated, playtested, and evaluated periodically, with alterations depending on whether the LO was achieved, and to what extent. Puzzles mechanics, levels, and other interactable mechanics were developed to target specific concepts related to the LOs, with each mechanic relevant in the level's solution, ensuring the player-learner's mastery in each concept.

A GBLE with the purpose as an assistive tool to achieving the selected learning outcomes was developed in the Unity game engine, with LO mechanics as a part of the core gameplay loop and several non-LO mechanics to support it. Additionally, gamification elements such as narrative, objectives, level progression and challenges were included in the GBLE to enhance the engagement and motivation of player-learners.

A pre-test and post-test comparison and a MEEGA+ survey were conducted to evaluate the GBLE's effects on supporting the achievement of the target LOs as well as the engagement of the player-learners. In order to compare the pre-test and post-test results, a paired t-test and a Wilcoxon signed-rank test were performed to gauge the GBLE's effectiveness in the support of the achievement of the target LOs.

The study results suggest that the game-based learning environment significantly supports the achievement of the selected target LOs of the player-learners, while keeping them engaged. The pre-test and post-test comparison results show that there is a significant difference after playtesting, with higher scores on the post-test. While some factors in the MEEGA+ survey received low scores, the average scores of all factors still classify the game's quality as excellent.

\section{Recommendations}
\label{sec:recommendations}
One of the things that Conic Lab could have improved on was achieving another gamification method for the players through the use of an achievement system. For example, when a player completes a level in the least moves possible, then they could receive a green highlight for the level in the main menu. This could improve the engagement/immersion of the players as they will receive a reward which would also signify that they were efficient in a level. 

The amount of playtesting could be improved to generate more ideas as to where the game could be more refined. In one group of playtesters, a lot of information and point for improvements have already been discovered such as adding music, improving the immersive experience when playing the game, and so on. More playtesting would allow Conic Lab to discover more areas to improve on and this could make the game better.

Lastly, a mobile phone version for the game would also be helpful. It would allow more room for students to play the game as more people have phones rather than personal computers. Accessibility would also be improved and this could lead to more playtesters.

Added Chapter 7





