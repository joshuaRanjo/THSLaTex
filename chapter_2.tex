%%%%%%%%%%%%%%%%%%%%%%%%%%%%%%%%%%%%%%%%%%%%%%%%%%%%%%%%%%%%%%%%%%%%%%%%%%%%%%%%%%%%%%%%%%%%%%%%%%%%%%
%
%   Filename    : chapter_2.tex 
%
%   Description : This file will contain your Review of Related Literature.
%                 
%%%%%%%%%%%%%%%%%%%%%%%%%%%%%%%%%%%%%%%%%%%%%%%%%%%%%%%%%%%%%%%%%%%%%%%%%%%%%%%%%%%%%%%%%%%%%%%%%%%%%%

\chapter{Review of Related Literature}
\label{sec:relatedlit}

\section{Serious Games}
Most games are created so that people can be engaged in the content of the game, with the primary purpose of entertaining its users. However, there are games that are not limited to just entertainment, and this genre of games is called a serious game. Serious games are defined as games that do not have entertainment, enjoyment, or fun as their primary purpose, and can be distinguished from video games by their design objectives (Eid, Laamarti, \& El Saddik, 2014). According to Gamelearn (2020), the primary objective of serious games is learning or practicing a skill rather than entertainment, and serious games have been already used in several sectors such as education, science and health, and even aeronautics. Serious games are games with a main goal of helping people learn and develop skills, this could even result in behavior change of the players. The things that serious games specialize in is that they are made to be entertaining, engaging, and immersive (Roolvink, 2021). They are designed to let players solve problems in different areas and involve them in challenges and rewards while maintaining the engaging and entertaining aspects that serious games have (Roolvink, 2021). According to Thalmann (2014), most researchers' definition of a serious game is that it must have an entertainment dimension, and that a serious game has a potential to provide an immersive experience to players through multimodal interaction in different contexts such as training, health, and education.
Serious games in different sectors have been implemented and are available in several platforms such as mobile and computer. Some examples of the popular existing serious games are Microsoft Flight Simulator (1982), World Without Oil (2007), and Sea Hero Quest (2016). Microsoft Flight Simulator is a flight simulator program created for computers, wherein players can learn and practice their piloting skills in a real-time atmospheric simulation. Second, World Without Oil was an alternate reality game that lasted for 32 days, wherein the main objective is to help its players visualize what subtle and deep effects an oil crisis has in their lives, in order to call attention to a possible near-future oil shortage. Lastly, Sea Hero Quest is a mobile game released in association with Alzheimer’s Research UK as a contribution to research in dementia. People with dementia can face challenges with navigation early on, and by playing the game, reliable data on how navigational abilities can change in the healthy brain across life is provided anonymously to researchers.

\section{Precalculus Pedagogy}
\subsection{Challenges in Precalculus Pedagogy}
According to Kauffmann, Archava, Castles.., (2011), studies indicated that STEM students in the US had a 30-50\% rate of retention in engineering, and these had diverse factors including learning styles and psychological dispositions. The key factors that remained an issue in the retention of students are the poor teaching and pedagogy. The authors were also able to find out in their research according to the students’ perceptions that high school preparation is a big issue because it appears that high school mathematics barely focused on polynomial and quadratic equations in a trade off for trigonometric and exponential functions. This led to a damaging impact in pre-calculus in engineering programs.
\subsection{Unique situation of Precalculus in Philippines}
In a study by Jaudinez (2019), a number of problems were identified when it comes to teaching mathematics to senior high school students of which precalculus is specified within the study. The identified problems include the general lack of teaching strategies for difficult topics, proper teaching resources and tools, and performance-based activities. Teaching is heavily dependent on the traditional face to face method, and difficult lessons are not reinforced. The teaching of SHS mathematics started in 2016 without any references provided, and the textbooks released only a year after were not appropriate for SHS students in terms of difficulty, complexity, and context. Lastly, the lessons in precalculus lack enhancement activities or tasks that are performance-based, and only practice exercises or tests are given to the students.
\subsection{Precalculus Pedagogy Frameworks in SHS}
When it comes to the pedagogies used in the curriculum created by DepEd, they are required to use pedagogical approaches that are constructivist, inquiry-based, reflective, collaborative and integrative. As stated by the context expert, DepEd provided a way to assess learners and to determine what should be expected of them. According to the DepEd Order 8, s. in 2015, schools shall consider three (3) basic assessment criteria: written works (WW),  performance tasks (PT), and quarterly assessments (QA). In terms of teaching strategies implemented, teachers have the freedom to select which strategy, technique, However, due to the on-going nature of the Global Pandemic of COVID-19, DepEd adapted the implementation of Blended Learning.
\subsection{Blended Learning during the COVID-19 Pandemic}
According to an official statement of DepEd on 8th of June 2020, a direct order from President Duterte was issued to postpone face-to-face classes and thus the groundwork for blender learning was paved. Making use of radio, television, online and modular learning which are pre-existing methods already but are now being prepared and updated to match the needs for the current situation. Meanwhile, teachers were also being introduced and trained on how to use newer platforms and innovative tools to help them adjust in teaching in a new environment.

\section{Game-based Learning}
Game-based learning is a trend that has been implemented in several different areas, one of these being education. According to Trybus (2014), game-based learning refers to the borrowing of certain gaming principles and applying them to real-life settings to engage its users. It is a type of game play with specific learning objectives and outcomes, with a design process that involves balancing the need to cover the subject matter with the desire to prioritize gameplay (Homer, Plass \& Kinzer, 2015). EdTechReview (2013) states that game-based learning is a type of game play that includes learning outcomes and is designed to have a balance between the ability of the player to apply the said subject matter to the real world and subject matter with gameplay.
Several game-based learning environments have been implemented in different domains in order to improve the learning achievement and motivation of students. In a study by Stiller and Schworm (2019), a game-based learning environment was implemented in the domain of science and english using a game called CellCraft in order to assess the effects of gameplay on motivation, cognitive load, and performance. Results of this study showed that gaming improved the knowledge of the students and reduced their cognitive load, enabling the students to focus on their learning. Another implementation of a game-based learning environment was done in the domain of American Sign Language (ASL) by Kamnardsiri, Hongsit, Khuwuthyakorn, and Wongta (2017) in order to explore the effectiveness of the usability of the intelligent game-based system for learning ASL. The results of this study showed a significant difference in the post-test scores for the game-based learning group and the group learning using the traditional face to face method, the scores of the gaming group being higher. In addition to this, the system also engaged the students to concentrate on ASL vocabularies and gave them a sense of control and concentration while playing the game.


















