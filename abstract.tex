%%%%%%%%%%%%%%%%%%%%%%%%%%%%%%%%%%%%%%%%%%%%%%%%%%%%%%%%%%%%%%%%%%%%%%%%%%%%%%%%%%%%%%%%%%%%%%%%%%%%%%
%
%   Filename    : abstract.tex 
%
%   Description : This file will contain your abstract.
%                 
%%%%%%%%%%%%%%%%%%%%%%%%%%%%%%%%%%%%%%%%%%%%%%%%%%%%%%%%%%%%%%%%%%%%%%%%%%%%%%%%%%%%%%%%%%%%%%%%%%%%%%

\begin{abstract}

During the 2018 and 2022 Programme for International Student Assessment (PISA) by the Organization for Economic Co-operation and Development (OECD), the Philippines placed low in the rankings, especially in mathematics. A subject in which focus and attention are needed, teaching mathematics to students seems to have become a challenging task in itself by not losing the interest of the students, a task in which video games usually excel. Video games have a long history of being used mainly for entertainment purposes, but the educational application of this technology is becoming well known as well. Educational games are the fruit of utilizing video game technology in pursuit of the improvement of the quality of education. Bringing out the positive benefits of video games to support a person’s learning development is the main goal of educational games. Game-based learning makes use of the positive benefits of educational games to support learning outcomes.  This paper aims to design, develop, and evaluate a game-based learning environment that could assist in teaching a Precalculus concept in the form of Conic Sections, a subject involving Analytical Geometry.

%
%  Do not put citations or quotes in the abract.
%

%Keywords can be found at \url{http://www.acm.org/about/class/class/2012?pageIndex=0}.  Click the 
%link ``HTML'' in the paragraph that starts with ''The \textbf{full CCS classification tree}...''.

\begin{flushleft}
\begin{tabular}{lp{4.25in}}
\hspace{-0.5em}\textbf{Keywords:}\hspace{0.25em} & Education, Game-Based Learning, Serious Games, Precalculus, etc.\\
\end{tabular}
\end{flushleft}
\end{abstract}
